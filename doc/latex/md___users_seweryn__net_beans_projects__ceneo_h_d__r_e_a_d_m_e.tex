Ceneo\+HD -\/ Projekt. Proces E\+TL


\begin{DoxyEnumerate}
\item Stwórz aplikację, która pozwoli na przeprowadzenie procesu E\+TL na który składa się\+: a. pozyskanie danych ze źródeł zewnętrznych, b. poddanie danych odpowiednim przekształceniom, c. zasilenie danymi bazy danych.
\item Celem działania aplikacji będzie pobranie z serwisu Ceneo.\+pl wszystkich opinii o produkcie, którego kod zostanie podany jako parametr. Dla uproszczenia działanie aplikacji można ograniczyć do produktów z działów Komputery, Fotografia oraz Telefony i akcesoria.
\item Dla każdej opinii należy pobrać\+: a. Wady produktu b. Zalety produktu c. Podsumowanie opinii d. Liczba gwiazdek e. Autor opinii (jeśli brakuje, to np. anonim) f. Data wystawienia opinii g. P\+O\+L\+E\+C\+AM / N\+IE P\+O\+L\+E\+C\+AM h. Ile osób uznało opinię za przydatną, a ile za nieprzydatną W przypadku, gdy brakuje którejś wartości pole w tabeli należy pozostawić puste.
\item Dla każdego produktu należy przechować w bazie danych\+: a. Rodzaj urządzenia b. Markę c. Model d. Dodatkowe uwagi
\end{DoxyEnumerate}
\begin{DoxyEnumerate}
\item Zadanie składa się z trzech części\+: a. Aplikacji b. Dokumentacji technicznej c. Dokumentacji użytkownika
\item Aplikację należy stworzyć w dowolnej, wybranej przez siebie technologii. Podobnie bazę danych.
\item Jeśli aplikacja ma formę aplikacji webowej należy umieścić ją na dowolnym serwerze (może to być serwer v-\/ie) tak, aby możliwy był do niej dostęp przez przeglądarkę internetową.
\item W przypadku aplikacji w innej formie niż webowa należy określić minimalne wymagania potrzebne do zainstalowania/uruchomienia aplikacji (środowisko, biblioteki itp.)
\item Aplikacja powinna mieć możliwość przeprowadzenia całego procesu E\+TL na raz (np. przycisk E\+TL) jak również przeprowadzenia każdej jego części oddzielnie – jedna po drugiej (np. trzy kolejne przyciski E, T, L).
\item W przypadku wykonywania procesu E\+TL etapami wykonanie kolejnego etapu powinno być możliwe tylko po uprzednim wykonaniu wcześniejszego etapu (np. nie powinno się dać wykonać operacji Transform, jeśli wcześniej nie została wykonana operacja Extract).
\item Po przeprowadzeniu pełnego procesu E\+TL lub ostatniego etapu Load produkty uboczne procesu (np. pobrane pliki) powinny być usuwane (tym samym po wykonaniu Load nie powinno być możliwości wykonania operacji Transform bez ponownego wykonania Extract).
\item Po przeprowadzeniu pełnego procesu E\+TL jak również po każdej z jego składowych powinny być wyświetlane informacje statystyczne np. ile plików zostało pobranych, ile rekordów zostało załadowanych do bazy danych.
\item Dane w bazie nie powinny się dublować – jeśli dana opinia zastała dodana do bazy przy poprzednim uruchomieniu procesu E\+TL nie powinna zostać dodana przy kolejnych. Podobnie w przypadku produktów.
\item Aplikacja powinna mieć przycisk pozwalający na wyczyszczenie w bazie danych dotyczących opinii tak, aby możliwe było zaobserwowanie procesu zasilenia bazy opiniami.
\item Aplikacja powinna umożliwiać wyświetlenie opinii z bazy w podziale na produkty, których opinie dotyczą.
\item Dodatkowo aplikacja powinna mieć możliwość eksportu danych do pliku .csv (osobny przycisk lub opcja) i oddzielnych plików tekstowych dla każdej opinii
\item Projekt powinien być uniwersalny – powinien nadal działać, gdy na stronie zostaną dodane nowe opinie lub jakaś opinia zostanie usunięta.
\item Projekt powinien zawierać dokumentację techniczną (traktowaną jako opis A\+PI), która może być napisana podobnie do dokumentacji klas bibliotek Javy1
\end{DoxyEnumerate}
\begin{DoxyEnumerate}
\item Dokumentacja techniczna powinna zawierać\+: a. Nazwy i wersje użytych technologii (języki programowania, sposób przechowywania danych). b. Projekt fizyczny bazy danych wykorzystanej w projekcie. c. Informacje na temat środowiska, wymaganego do uruchomienia programu. d. Linki do oprogramowania, które to środowisko tworzą. e. (jeśli obiektowo) nazwy klas, tabelkę z listą i opisem atrybutów, tabelkę z listą i opisem (wartość zwracana, przyjmowane parametry, opis działania) funkcji składowych (z konstruktorami i destruktorami włącznie); f. (jeśli strukturalnie)\+: tabelkę z listą i opisem zmiennych, tabelkę z listą i opisem (wartość zwracana, przyjmowane parametry, opis działania) funkcji.
\item Ponadto projekt powinien zawierać dokumentację użytkownika – instrukcje obsługi programu przedstawiającą krok po kroku (najlepiej ze zrzutami ekranu) jak z aplikacji korzystać począwszy od jej „instalacji” (sposobu uruchomienia) aż do wyświetlenia danych z bazy pobranych w trakcie działania aplikacji
\end{DoxyEnumerate}
\begin{DoxyEnumerate}
\item Przez platformę e-\/learningową należy przesłać\+: a. Link do publicznego repozytorium Git z kodem aplikacji! b. Źródło danych określające strukturę przechowywanych danych wraz z przykładowymi danymi (np. kod S\+QL) c. Link do aplikacji jeśli ma ona postać aplikacji webowej d. Link do dokumentacji technicznej lub jej kod jeśli ma ona postać aplikacji e. Dokumentację techniczną oraz dokumentację użytkownika złożone w jeden spójny dokument tekstowy
\item Jako stronę tytułową dokumentacji wykorzystaj plik dostępny na platformie elearningowej. Wypełnij tabelkę danymi członków zespołu. Podaj za jakie czynności odpowiadał każdy członek zespołu.
\item Plikowi z dokumentacjami nadaj nazwę według wzoru\+: Numer\+Grupy\+Projektowej \+\_\+\+Projekt.\mbox{[}doc$\vert$docx\mbox{]}
\item Prześlij pliki i linki przez platformę e-\/learningową w terminie wskazanych we właściwym zadaniu (aktywności). Wystarczy jeśli jedna osoba z grupy prześle pliki projektu.
\item Wydrukowany dokument z instrukcjami przynieś na konsultacje na których zespół będzie prezentował i zaliczał projekt. 
\end{DoxyEnumerate}